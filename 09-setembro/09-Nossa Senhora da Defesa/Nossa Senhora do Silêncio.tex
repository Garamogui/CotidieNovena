\documentclass[a4paper,14pt]{extarticle} \usepackage[utf8]{inputenc}
\usepackage[T1]{fontenc}
\usepackage[margin=2.5cm]{geometry}

% Fonte Caladea se existir, senão lmodern
\IfFileExists{caladea.sty}{
  \usepackage{caladea}
}{
  \usepackage{lmodern} }
\usepackage{ragged2e}
\usepackage{graphicx}
\usepackage[portuguese]{babel}
\usepackage{wrapfig}
\usepackage{hyperref}
\usepackage{fancyhdr}
\usepackage{xcolor}
\usepackage{rotating}
\usepackage{titlesec}
\usepackage{epigraph}
\usepackage{dirtytalk}
\usepackage{indentfirst} % Indenta o primeiro parágrafo após seções

% Ajuste do recuo de parágrafo
\setlength{\parindent}{1.5em}

% Centralizar títulos
\titleformat{\section}
  {\normalfont\centering\bfseries\Large}{\thesection}{1em}{}

\titleformat{\subsection}
  {\normalfont\centering\bfseries\large}{\thesubsection}{1em}{}

\titleformat{\subsubsection}
  {\normalfont\centering\bfseries}{\thesubsubsection}{1em}{}

% -------------- Símbolos de Versículo e Resposta --------------
% Definição do símbolo (a “barrinha” inclinada)
\makeatletter
\newcommand{\vers@resp@sym}{%
  \raisebox{0.2ex}{\rotatebox[origin=c]{-20}{$\m@th\rceil$}}%
}
% macro interna que sobrepõe a barrinha e a letra V ou R
\newcommand{\vers@resp}[2]{%
  {\ooalign{%
     \hidewidth\kern#1\vers@resp@sym\hidewidth\cr
     #2\cr
  }}%
}
% comandos públicos \versicle e \response
\DeclareRobustCommand{\versicle}{\vers@resp{-0.1em}{V}}
\DeclareRobustCommand{\response}{\vers@resp{0pt}{R}}
\makeatother
% ^------------- Símbolos de Versículo e Resposta -------------^

% Rodapé com imagem e página
\pagestyle{fancy}
% ---- Cabeçalho ------------
\fancyhf[C]{}
% ----- Rodapé --------------
\fancyfoot[LO,LE]{%
  \includegraphics[scale=0.2]{assets/cross.png}\quad
  \textit{Novena a \textbf{Nossa Senhora da Defesa}}
}
\fancyfoot[RO,RE]{\thepage}

\begin{document}


\begin{center}
  {\huge Novena a Nossa Senhora da Defesa}
\end{center}

\say{
Oh! Nossa Senhora da Defesa, virgem poderosa, recorro a Vossa proteção contra todos os assaltos do inimigo, pois Vós sois o terror das forças malignas. Eu seguro no Vosso manto santo e me refugio debaixo dele para estar guardado, seguro e protegido de todo mal. Mãe Santíssima, Refúgio dos pecadores, Vós recebestes de Deus o poder para esmagar a cabeça da serpente infernal e com a espada levantada afugentar os demônios que querem acorrentar os filhos de Deus. Curvado sobre o peso dos meus pecados venho pedir a Vossa proteção hoje e em cada dia da minha vida, para que vivendo na luz do vosso filho, Nosso Senhor Jesus Cristo eu possa depois desta caminhada terrena, entrar na pátria celeste. Amém!
}

\par\noindent\rule{\textwidth}{0.4pt}

\tableofcontents
\thispagestyle{empty}

% --- Vida / Origem da Novena ---
\newpage

\section{História da Devoção}

Segundo o que se sabe, Nossa Senhora já tinha intervindo milagrosamente em Cortina d’Ampezzo em 512, segundo os relatos dessa época, os habitantes se reuniram e puseram-se a rezar suplicando a ajuda da Mãe de Deus, decididos foram defender seu território, infelizmente foi-se percebido a inevitável invasão lombarda, os ampezzanos invocaram o nome da Santa Mãe de Deus. Ela então teria aparecido, portando uma espada, sobre as nuvens que ali passava. Quando os lombardos tentaram invadir a cidade, Nossa Senhora teria envolvidos os seus inimigos com as nuvens impedindo-lhes a visão. Por estarem as cegas acabaram por lutar entre si até se derrotarem. Através dessa aparição surgiu a devoção de Nossa Senhora da Defesa.

Nove séculos, precisamente no ano de 1412 ocorreu a sua segunda aparição: Em Cortina d’Ampezzo já havia uma capela dedicada a Nossa senhora da Defesa desde o século XIV, mas apenas em 1412, ocorreu novamente uma intervenção miraculosa da Virgem Maria. Novamente a cidade vinha sendo ameaçada por mais uma invasão, para piorar os seus habitantes estavam quase que desarmados, o inverno daquela ano foi muito rigoroso e os godos estavam para invadir a cidade. Assim como em sua primeira aparição os habitantes se puseram a rezar invocando o seu santo nome, os ampezzanos não tinham um treinamento militar qualificado para lidar com aquela situação, e nem mesmo outros recursos bélicos que fosse capaz de vencer um poderoso inimigo, a solução encontrada foi permanecer unido, em oração, esperando a interseção de Nossa Senhora.

O exército Gogo logo impetrou o seu ataque, os habitantes unidos intensificaram as suas orações e rezar com fervor. Não durou muito, logo Nossa Senhora apareceu sobre as nuvens. dessa vez portando uma espada de fogo em sua mão direita e segurando o Menino Jesus com o braço esquerdo. Ela desceu sobre o local onde estava ocorrendo um terrível massacre. Assim como em sua primeira aparição novamente as nuvens se envolveram aos seus inimigos, os cegando e atordoando-os, confundidos passaram a guerrear contra si mesmos até que, por fim, os godos se destruíram. Os italianos salvos do perigo, agradecidos pela vitória e impressionados com o novo sinal milagroso de Nossa Senhora da Defesa passaram a festejar o ocorrido até os dias de hoje.

A veneração a Nossa Senhora por meio desse título iniciou-se primeiramente na Catedral de Ozieri, em Sassari, na Itália, ainda nesta mesma região ela é a Padroeira da cidade de Stintino, onde se comemora e celebra a sua festa, em 18 de setembro, mas desde 1337 já havia devoção a Nossa Senhora da Defesa, principalmente por parte da Confraria dos Battuti (confraria de leigos católicos).

Um Santuário em honra de Nossa da Defesa foi construído em 1750. em sua fachada, acima da porta principal, tem um quadro com uma pintura de Nossa Senhora da Defesa representando o milagre de 512, Em seu altar-mor se encontra sua imagem, considerada a mais antiga produzida no século XIV e é ricamente vestida e coroada. No dia 18 de setembro, é comum em seu santuário fazer uma procissão noturna a luz de velas, podendo ser por terra ou mar, reunindo milhares de devotos da Europa e do mundo. Na Catedral da cidade de Ozieri existe um altar para sua devoção até hoje. 


% --- Orações Diárias ---
\newpage

\section{Novena Nossa Senhora da Defesa}

\subsection{1° Dia}

Ó Santíssima Virgem Maria, que aparecestes de maneira milagrosa com a espada na mão, para defender os moradores de Ampezzano, nós vos suplicamos que venhais com vossa espada nos defender dos assaltantes e sequestradores, por todos os caminhos por onde nós andamos e dai-nos a graça de vivermos a palavra de Deus. Amém. 

Pai Nosso... Ave Maria... Glória ao Pai... Salve Rainha...  

Oração de N.S. da Defesa  (todos os dias)

Óh! Nossa Senhora da Defesa, virgem poderosa, recorro a Vossa proteção contra todos os assaltos do inimigo, pois Vós sois o terror das forças malignas. Eu seguro no Vosso manto santo e me refugio debaixo dele para estar guardado, seguro e protegido de todo mal. Mãe Santíssima, Refúgio dos pecadores, Vós recebestes de Deus o poder para esmagar a cabeça da serpente infernal e com a espada levantada afugentar os demônios que querem acorrentar os filhos de Deus. Curvado sobre o peso dos meus pecados, venho pedir a Vossa proteção hoje e em cada dia da minha vida, para, que vivendo na luz do Vosso Divino Filho, Nosso Senhor Jesus Cristo eu possa depois desta caminhada terrena, entrar na pátria celeste. Amém!


\subsection{2° Dia}

Virgem Poderosa, nossa Mãe Celeste, nós Vos pedimos, defendei-nos contra todo tipo de violência que possa nos atingir, à nossa casa, à nossa família, e dai-nos a graça de vivermos em caridade. Amém. 

Rezar a \textbf{\nameref{ladainha}} e \textbf{\nameref{oracao-final}}


\subsection{3° Dia}

Ó Nossa Senhora da Defesa, Vós sois a defensora da juventude. Por isso, nós Vos pedimos que defendais nossos jovens contra o terror das drogas que estão destruindo lares, famílias e vidas, que são preciosas para a construção de um mundo melhor; também Vos pedimos que defendais a vida no ventre materno, inocente e indefesa, fomentando o amor dos pais ou as adoções, e dai-nos a graça de vivermos na esperança. Amém.

Rezar a \textbf{\nameref{ladainha}} e \textbf{\nameref{oracao-final}}


\subsection{4° Dia}

Nossa Senhora, Mãe dos aflitos, socorrei-nos e defendei-nos contra o desemprego, contra a fome e a miséria nestes tempos tão difíceis e dai-nos a graça de vivermos cada dia na força do amor e do perdão. Amém.

Rezar a \textbf{\nameref{ladainha}} e \textbf{\nameref{oracao-final}}


\subsection{5° Dia}

Nossa Senhora, refúgio dos pecadores, muitas vezes nos deixamos levar pela dúvida e nos enveredamos por caminhos que nos trazem sofrimentos e desilusões. A Vós recorremos para pedir a Vossa defesa contra toda incredulidade, indiferença e infidelidade que vier de encontro a nós e dai-nos a graça de perseverarmos nesta igreja Católica Apostólica Romana onde nascemos e cuja fé queremos sempre professar. Amém.

Rezar a \textbf{\nameref{ladainha}} e \textbf{\nameref{oracao-final}}


\subsection{6° Dia}

Nossa Senhora da Defesa, saúde dos enfermos, nós nos colocamos aos Vossos pés para pedir Vossa defesa contra todos os males do corpo, males da mente e males do espírito. Pedimos a graça de vivermos confiantes na ação de Deus em nossas vidas. Amém.

Rezar a \textbf{\nameref{ladainha}} e \textbf{\nameref{oracao-final}}


\subsection{7° Dia}

Ó Nossa Senhora da Defesa, Mãe de Deus e nossa Mãe, quantas vezes neste caminhar da vida terrena, nos sentimos atribulados por tantos problemas que chegam até nós por causa de forças negativas e malignas. Suplicamos vossa defesa contra toda inveja, vaidade e ódio que vier de encontro a nós, nossa casa, nosso trabalho e nossa família. Dai-nos a graça da fidelidade a Deus. Amém.

Rezar a \textbf{\nameref{ladainha}} e \textbf{\nameref{oracao-final}}


\subsection{8° Dia}

Virgem Poderosa, nós nos colocamos na Vossa presença para implorar Vossa defesa contra as catástrofes que possam se abater sobre nós. Contamos com Vossa defesa e proteção para nós, nossa cidade, nosso País, para o mundo inteiro, e dai-nos a graça de aceitarmos a vontade de Deus em todas as situações. Amém.

Rezar a \textbf{\nameref{ladainha}} e \textbf{\nameref{oracao-final}}


\subsection{9° Dia}

Nossa Senhora da Defesa, Rainha da Paz, humildemente vimos hoje clamar Vosso auxilio para que venhais em nossa defesa contra a guerra e destruição que ameaça a Humanidade e pedimos a graça de vivermos na união e na paz. Amém.

Rezar a \textbf{\nameref{ladainha}} e \textbf{\nameref{oracao-final}}




\section{Oração Final} \label{oracao-final}
\[
  \textbf{Pai-Nosso, Ave-Maria, Glória Ao Pai.}
\]

\response. \quad Roga por nós, Santa Mãe de Deus!

\versicle. \quad Para que nos tornemos dignos das promessas de Cristo. \\

\textbf{Oremos:} Oh! Nossa Senhora da Defesa, Virgem poderosa, recorro à Vossa proteção contra todos os assaltos do inimigo, pois Vós sois o terror das forças malignas.  

Eu seguro no Vosso manto santo e me refugio debaixo dele para estar guardado, seguro e protegido de todo mal.  

Mãe Santíssima, Refúgio dos pecadores, Vós recebestes de Deus o poder para esmagar a cabeça da serpente infernal e com a espada levantada afugentar os demônios que querem acorrentar os filhos de Deus.  

Curvado sobre o peso dos meus pecados, venho pedir a Vossa proteção hoje e em cada dia da minha vida, para que, vivendo na luz do Vosso Filho, Nosso Senhor Jesus Cristo, eu possa, depois desta caminhada terrena, entrar na Pátria Celeste. Amém!  


\section{Ladainha} \label{ladainha}

Nossa Senhora da Defesa, vós que sois o terror das forças malignas, 

- Da inveja, vaidade e ódio, \textbf{defendei-nos.}

- Do orgulho, soberba, vanglória e atrevimento,...

- Da arrogância, presunção, cinismo e autossuficiência,...

- Da hipocrisia, fingimento e deboche,...

- Do egoísmo e ganância,...

- Do preconceito, julgamento e discriminação,...

- Da preguiça, acomodação, má vontade e omissão,...

- Da derrota fracasso e desânimo,... 

- Da depressão, aflição e desespero,...

- Da prostração, alienação e indolência,...

- Da autopiedade, da auto-rejeição e da auto-condenação,...

- Dos complexos, da carência afetiva e emocional,...

- Dos problemas de ordem sexual,...

- Da superproteção e dependência emocional,...

- Da adulação e idolatria a coisas e pessoas,...

- Da dominação, ciúmes, posse, opressão e perseguição,...

- Da murmuração e impaciência,...

- Da rejeição, desprezo e abandono,... 

- Do isolamento e timidez,...

- Da solidão, melancolia e angústia,...

- Do suicídio, desejo de morrer e confusão mental,...

- Da culpa e falta de perdão,...

- Da acusação, da quebra de segredo e da avareza,...

- Da mentira, calúnia e falso testemunho,...

- Da condenação, crueldade e fofoca,...

- Da incoerência e escândalo,...

- Da ingratidão, falsidade e cilada,...

- Dos assaltos, sequestros, vandalismo, tráfico e drogas,...

- Das catástrofes, calamidades e pestes,...

- Da miséria, fome, dívidas e desemprego,...

- Do medo, insegurança e maus pensamentos,...

- Da preocupação exagerada, insônia e tensão nervosa,...

- Do pânico, pavor e susto,...

- Do exagero e descontrole,...

- Da ansiedade e inquietação,...

- Dos males físicos, psíquicos e espirituais,...

- Das maldições de antepassados e malefícios,...

- Das manias, taras e vícios,...

- Da malícia, impureza e imoralidade,...

- Da sedução, orgia e fornicação,...

- Do adultério, prostituição e promiscuidade,...

- Da compulsividade para ter, ser, poder, e prazer,...

- Da compulsividade no jogo e dependência química,...

- Da falsa oração, do ocultismo, dos objetos contaminados e superstição...

- Da injustiça, revolta e frustração,...

- Da destruição e desejo da vingança,...

- Do assassinato e desejo da morte de alguém,...

- Da revolta contra Deus,...

- Da descrença, dúvida e falta de fé,...

- Da incredulidade, da infidelidade e indiferença,...

- Da desobediência, zombaria e gozação,...

- Dos sacrilégios e blasfêmias,...

- Da falta de respeito com as coisas de Deus,...

- Da não aceitação de Jesus Cristo como nosso único “Senhor e Salvador”,...  

\vspace{2em}

\textbf{Oremos:} Concedei a Vossos servos, nós Vo-lo pedimos, Senhor Deus, que possamos sempre gozar da saúde da alma e do corpo e, pela gloriosa intercessão da bem-aventurada Virgem Maria, Nossa Senhora da Defesa, sejamos livres da tristeza e alcancemos a eterna alegria. Por Cristo Nosso Senhor. Amém.   


\vfill

\begin{center}
\subsection*{Fontes:}
Adaptado de: \underline{\href{https://precantur.blogspot.com/2020/01/Nossa-Senhora-da-Defesa.html}{Thesaurus Precum}} e \underline{\href{https://pt.wikipedia.org/wiki/Nossa_Senhora_da_Defesa}{Wikipédia}}
\end{center}


\end{document}
