\documentclass[18pt]{article}
\usepackage[utf8]{inputenc}
\usepackage[T1]{fontenc}
\usepackage{ragged2e}
\usepackage{caladea}
\usepackage{graphicx}
\usepackage{longtable}
\usepackage{wrapfig}
\usepackage{rotating}
\usepackage{epigraph}
\usepackage[normalem]{ulem}
\usepackage{hyperref}
\usepackage{amsmath}
\usepackage{amssymb}
\usepackage{capt-of}
\usepackage{hyperref}
\usepackage{fancyhdr}

\title{
   NOVENA DO ABANDONO A JESUS}
\date{Início da Novena: 01/03}

% Comando para fazer "Sumário" não aparecer no Sumário.
\renewcommand{\contentsname}{Sumário}
\begin{document}
\maketitle

\thispagestyle{empty} %zera a primeira página

\pagestyle{fancy}
\fancyhf{} % clear existing header/footer entries
\fancyfoot[LO, CE]{
  \includegraphics[scale=0.2]{./assets/cross.png} Ó Jesus, eu me abandono ao Senhor!
}
% Place Page X of Y on the right-hand
% side of the footer
\fancyfoot[R]{\thepage}

\newpage

\tableofcontents

\centering
\vfill
Visite-nos no Telegram: \url{https://t.me/CotidieNovena}
\newpage

\newpage


\begin{justify}

 \begin{center}
  \section{Apresentação}\label{sec:Apresentação} % (fold)
 \end{center}


 Dolindo Ruotolo, um frade capuchinho que viveu de 1882 a 1970, compreendeu profundamente a relação entre nossa necessidade e a bondade de Deus. Ordenado aos 23 anos, Dolindo passou a vida em oração, sacrifício e serviço. Ele ouviu confissão, deu orientação espiritual e cuidou dos necessitados. Por um tempo, serviu como diretor espiritual de Padre Pio. Inclusive, quando alguns peregrinos de Nápoles, onde residia Dolindo, iam para Pietrelcina, Padre Pio costumava dizer: “Por que vocês veem aqui, se vocês têm Dom Dolindo em Nápoles? Vão até ele, ele é um santo!”

 O frade tornou-se conhecido por sua espiritualidade de rendição. Bem consciente da fraqueza e da necessidade humanas, Dolindo viu isso como uma forma de promover uma união contínua com Deus. Ao nos convidar a levar continuamente nossas preocupações ao Senhor, ele nos ensina que o foco de nossas orações não deve permanecer em nossas necessidades. Ele nos encoraja a levar nossas necessidades a Deus, deixando-o livre para cuidar de nós em sua sabedoria. Dolindo nos diz que o Senhor prometeu assumir plenamente todas as necessidades que confiamos a ele.

 Nas palavras de Jesus a Dolindo: “Por que você se confunde com a sua preocupação? Deixe o cuidado de seus assuntos para mim e tudo ficará em paz. Digo-lhe, na verdade, que todos os atos de entrega verdadeira, cega e completa produzem o efeito que você deseja e resolvem todas as situações difíceis. (…) Mil orações não são iguais a um ato de abandono; nunca esqueça isso. Não há melhor novena do que esta: ó Jesus, eu me abandono ao Senhor. Jesus, assuma o controle.”

\end{justify}

%%%%%%%%%%%%%%%%%%%%%%%%%%%%%%%%%%%%% Orações  %%%%%%%%%%%%%%%%%%%%%%%%%%%%%%%%%%%%%%%%%%%

\newpage
\section{Orações}\label{sec:Orações} % (fold)
\subsection{Oração Inicial}\label{sec:Oração_Inicial} % (fold)

Querido Senhor, agradecemos pelas muitas práticas espirituais e devoções que o Senhor deu à Sua Igreja para nossa salvação. Por favor, conceda graça a todas as pessoas através da devoção ao Abandono a Jesus!

\subsection{Primeiro Dia}
\textbf{\nameref{sec:Oração_Inicial}}

Por que vocês se confundem preocupando-se? Deixe o cuidado de seus negócios comigo e tudo ficará em paz. Eu digo a você com verdade que todo ato de entrega verdadeira, cega e completa a mim produz o efeito que você deseja e resolve todas as situações difíceis.

\textbf{\nameref{sec:Oração_Final}}

\subsection{Segundo Dia}
\textbf{\nameref{sec:Oração_Inicial}}

Render-se a mim não significa se preocupar, ficar chateado ou perder a esperança, nem significa me oferecer uma oração preocupada pedindo-me para segui-lo e transformar sua preocupação em oração. É contra essa rendição, profundamente contra ela, preocupar-se, ficar nervoso e desejar pensar nas consequências de qualquer coisa. É como a confusão que as crianças sentem quando pedem à mãe para cuidar de suas necessidades e, em seguida, tentam cuidar dessas necessidades por si mesmas, de modo que seus esforços infantis atrapalhem a mãe.

\textbf{\nameref{sec:Oração_Final}}

\subsection{Terceiro Dia}
\textbf{\nameref{sec:Oração_Inicial}}

Quantas coisas eu faço, quando a alma, em tanta necessidade espiritual e material, se volta para mim, me olha e me diz; Jesus, cuida disso; fechando os olhos e descansando. Com dor, você ora para que eu aja, mas que eu aja da maneira que você deseja. Você não se dirige a mim, em vez disso, quer que eu me adapte às suas ideias.

\textbf{\nameref{sec:Oração_Final}}

\subsection{Quarto Dia}
\textbf{\nameref{sec:Oração_Inicial}}

Você vê o mal crescendo ao invés de enfraquecer? Não se preocupe, feche os olhos e diga-me com fé: seja feita a tua vontade, cuide dela. Digo-lhe que cuidarei dela e que intervirei como um médico e realizarei milagres quando forem necessários.

\textbf{\nameref{sec:Oração_Final}}

\subsection{Quinto Dia}
\textbf{\nameref{sec:Oração_Inicial}}

E quando devo conduzi-lo por um caminho diferente do que você vê, eu o prepararei; Vou carregá-lo em meus braços; Vou deixar você se encontrar, como uma criança que adormece nos braços da mãe, na outra margem do rio. O que o perturba e o magoa imensamente são sua razão, seus pensamentos e preocupação, e seu desejo a todo custo de lidar com o que o aflige.

\textbf{\nameref{sec:Oração_Final}}

\subsection{Sexto Dia}
\textbf{\nameref{sec:Oração_Inicial}}

Você está insone; você quer julgar tudo, dirigir tudo e cuidar de tudo e se render à força humana, ou pior - aos próprios homens, confiando na sua intervenção - isso é o que atrapalha minhas palavras e meus pontos de vista. Oh, quanto desejo de ti esta rendição, de te ajudar e como sofro ao te ver tão agitado!

\textbf{\nameref{sec:Oração_Final}}

\subsection{Sétimo Dia}
\textbf{\nameref{sec:Oração_Inicial}}

Eu realizo milagres em proporção à sua entrega total a mim e ao fato de você não pensar em si mesmo. Semeio tesouros de graças quando você está na mais profunda pobreza. Nenhuma pessoa de razão, nenhum pensador, jamais realizou milagres, nem mesmo entre os santos.

\textbf{\nameref{sec:Oração_Final}}

\subsection{Oitavo Dia}
\textbf{\nameref{sec:Oração_Inicial}}

Feche os olhos e deixe-se levar pela corrente que flui da minha graça; feche os olhos e não pense no presente; desvie seus pensamentos do futuro da mesma forma que faria com a tentação. Repouse em mim, acredite na minha bondade, e prometo-lhe, com meu amor, que se você disser: Jesus, cuide disso, Eu cuidarei de tudo.

\textbf{\nameref{sec:Oração_Final}}

\subsection{Nono Dia}
\textbf{\nameref{sec:Oração_Inicial}}

Ore sempre com prontidão para se render, e você receberá dela grande paz e grandes recompensas, mesmo quando eu conferir a você a graça da imolação, do arrependimento e do amor. Então, qual sofrimento importa? Parece impossível para você? Feche os olhos e diga com toda a alma: Jesus, cuide disso.


\subsection{Oração Final}\label{sec:Oração_Final} % (fold)
Senhor, por esta novena, vos pedimos especialmente: 

\textit{(mencione suas intenções aqui).}

Ó Jesus, eu me abandono ao Senhor. Jesus, assuma o controle. (10 vezes)

\textbf{Pai Nosso, Ave Maria, Glória ao Pai.}

\subsection*{Créditos:}
\href{https://pocketterco.com.br/terco/novena-do-abandono-a-jesus}{Pocket Terço}

\end{document}
