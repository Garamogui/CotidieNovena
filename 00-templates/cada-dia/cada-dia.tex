% === Template 1 (ajustado para TOC completo) ===
\documentclass[18pt]{article}

\usepackage[utf8]{inputenc}
\usepackage[T1]{fontenc}
\usepackage{ragged2e}
\usepackage{caladea}
\usepackage{graphicx}
\usepackage{hyperref}
\usepackage{fancyhdr}

\author{Adaptado de publicação revisada}
\date{} % data fixa ou móvel, se desejar

% Rodapé com imagem e página
\pagestyle{fancy}
\fancyhf{}
\fancyfoot[LO, CE]{%
  \includegraphics[scale=0.2]{assets/cross.png}\quad
  \textit{Novena a \textbf{<Nome do Santo>}}
}
\fancyfoot[R]{\thepage}

\renewcommand{\contentsname}{Sumário}

\begin{document}

\tableofcontents
\thispagestyle{empty}

% --- Vida / Origem da Novena ---
\newpage
\section*{Vida / Origem da Novena}
\addcontentsline{toc}{section}{Vida / Origem da Novena}
\begin{justify}
  % Aqui você conta a vida ou origem da devoção
  [Escreva aqui uma breve biografia ou origem da novena.]
\end{justify}

% --- Oração Inicial (opcional) ---
\newpage
\section*{Oração Inicial}
\addcontentsline{toc}{section}{Oração Inicial}
\begin{justify}
  % Texto da oração inicial, se houver
  [Insira aqui a oração que abre a novena.]
\end{justify}

% --- Orações Diárias ---
\newpage
\section{Novena a \textbf{<Nome do Santo>}}
% seção já numerada, entra automaticamente no TOC
\begin{justify}

% Dia 1
\subsection{Dia 1}
[Insira aqui a oração específica do Dia 1.]

% Dia 2
\subsection{Dia 2}
[Insira aqui a oração específica do Dia 2.]

% …

% Dia 9
\subsection{Dia 9}
[Insira aqui a oração específica do Dia 9.]

\end{justify}

% --- Oração Final ---
\newpage
\section*{Oração Final}
\addcontentsline{toc}{section}{Oração Final}
\begin{justify}
  % Oração de encerramento
  [Insira aqui a oração final, ex.: Pai-Nosso, Ave-Maria e Glória.]
\end{justify}

\end{document}
