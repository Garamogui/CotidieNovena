\documentclass[18pt]{letter}

\usepackage[utf8]{inputenc}
\usepackage[T1]{fontenc}
\usepackage{ragged2e}
\usepackage{caladea}
\usepackage{longtable}
\usepackage{wrapfig}
\usepackage{rotating}
\usepackage{epigraph}
\usepackage[normalem]{ulem}
\usepackage{hyperref}
\usepackage{amsmath}
\usepackage{amssymb}
\usepackage{capt-of}
\usepackage{fancyhdr}
\usepackage{titlesec}


% definição do símbolo (a “barrinha” inclinada)
\newcommand{\vers@resp@sym}{%
  \raisebox{0.2ex}{\rotatebox[origin=c]{-20}{$\m@th\rceil$}}%
}

% macro interna que sobrepõe a barrinha e a letra V ou R
\newcommand{\vers@resp}[2]{%
  {\ooalign{%
     \hidewidth\kern#1\vers@resp@sym\hidewidth\cr
     #2\cr
  }}%
}

% comandos públicos \versicle e \response
\DeclareRobustCommand{\versicle}{\vers@resp{-0.1em}{V}}
\DeclareRobustCommand{\response}{\vers@resp{0pt}{R}}
\makeatother

%\date{Data móvel: 12 de julho}
\author{Adaptado de publicação revisada}

\renewcommand{\contentsname}{Sumário}

\begin{document}

\tableofcontents

\thispagestyle{empty}
\pagestyle{fancy}
\fancyhf{}
\fancyfoot[LO, CE]{
\includegraphics[scale=0.2]{./assets/cross.png} São João Gualberto, rogai por nós!
}
\fancyfoot[R]{\thepage}

\centering

\vfill

%%%%%%%%%%%%%%%%%%%%%%%%%%%%%%%%%%%% Introdução %%%%%%%%%%%%%%%%%%%%%%%%%%%%%%%%%%%%%%%%%%

\newpage

\begin{justify}
  \begin{center}
    \section{São João Gualberto}
  \end{center}

São João Gualberto foi um abade italiano, fundador da Ordem de Vallombrosa, venerado por sua coragem no perdão e na busca da paz. Sua vida foi marcada pelo exemplo do perdão de inimigos e pela defesa da Igreja contra a corrupção. Com sua intercessão, rogamos a Deus para que aprendamos a retribuir o mal com o bem e alcançar a verdadeira paz.

\end{justify}

%%%%%%%%%%%%%%%%%%%%%%%%%%%%%%%%%%%% Oração Inicial %%%%%%%%%%%%%%%%%%%%%%%%%%%%%%%%%%%%%%%%%%

\newpage

\begin{justify}
\begin{center}
  \section{Oração Inicial}\label{sec:OraçãoInicial}
\end{center}

Que a intercessão do bem-aventurado abade João nos recomende a Ti, suplicamos, Senhor, para que aquilo que não merecemos por nossos próprios méritos, alcancemos por seu patrocínio. Por nosso Senhor Jesus Cristo, Vosso Filho, que convosco vive e reina, na unidade do Espírito Santo, Deus, por todos os séculos dos séculos.

Amém.

Deus onipotente e eterno, fonte de paz e amante da concórdia, conhecer-Te é viver, servir-Te é reinar; estabelece-nos em Teu amor, para que, pelo exemplo do bem-aventurado abade João Gualberto, saibamos retribuir o mal com o bem e bênçãos pelas maldições, e assim alcancemos de Ti perdão e paz.

Amém.

Deus onipotente e eterno, fonte da verdadeira paz e amante da concórdia, conhecer-Te é a verdadeira vida, servir-Te é a perfeita liberdade. Estabelece-nos no Teu amor, para que, pelo exemplo do bem-aventurado João, Teu abade, saibamos retribuir o mal com o bem e bênçãos em vez de maldições, e assim encontremos em Ti perdão e paz. Por Jesus Cristo nosso Senhor.

Amém.
\end{justify}

%%%%%%%%%%%%%%%%%%%%%%%%%%%%%%%%%%%% Meditação e Preces %%%%%%%%%%%%%%%%%%%%%%%%%%%%%%%%%%%%%%%%%%

\newpage

\begin{justify}
\begin{center}
  \section{Meditação e Preces}
\end{center}

Ó verdadeiro discípulo da Nova Lei, que soubeste poupar um inimigo por amor à Santa Cruz! Ensina-nos a praticar, como tu, as lições transmitidas pelo instrumento de nossa salvação, que então se tornará para nós, como para ti, uma arma sempre vitoriosa sobre as forças do inferno. Poderíamos contemplar a Cruz e ainda recusar perdoar uma ofensa de nosso irmão, quando o próprio Deus não só esquece nossas graves faltas contra Sua Soberana Majestade, mas também morreu no madeiro para expiá-las?

O perdão mais generoso que uma criatura pode conceder é apenas uma pálida sombra do perdão que diariamente recebemos de nosso Pai celestial. Assim, o Evangelho que a Igreja canta em tua honra pode-nos ensinar que o amor aos inimigos é a mais próxima semelhança que podemos ter com nosso Pai celestial e o sinal de que somos verdadeiramente Seus filhos.

Tu possuías, ó João, este grande traço de semelhança. Aquele que, em virtude de Sua geração eterna, é o verdadeiro Filho de Deus por natureza, reconheceu em ti a marca de nobreza que te fez seu irmão. Quando Ele inclinou Sua sagrada Cabeça para ti, saudou em ti o caráter de filho de Deus, que havias acabado de manter tão belamente: um título mil vezes mais glorioso que os de tua nobre ascendência.

Que semente poderosa o Espírito Santo plantava naquele momento em teu coração! E quão ricamente Deus recompensa um único ato generoso! Tua santificação, a participação gloriosa que tiveste nas vitórias da Igreja, a fecundidade pela qual vives ainda hoje na Ordem que de ti nasceu: todas essas graças escolhidas para tua alma e para tantas outras dependiam daquele momento decisivo. O destino, ou a Justiça de Deus, como diriam teus contemporâneos, colocou teu inimigo em teu poder: como o tratarias? Ele merecia a morte; e naqueles dias cada homem era seu próprio vingador. Se tivesses então infligido o castigo devido, tua reputação teria aumentado, não diminuído. Terias obtido a estima de teus companheiros; mas a única glória que vale diante de Deus, e que também dura aos olhos dos homens, jamais teria sido tua. Quem te conheceria hoje? Quem sentiria a admiração e gratidão que teu nome inspira nos filhos da Igreja?

O Filho de Deus, vendo que tuas disposições eram conformes às de Seu Sagrado Coração, encheu-te de Seu próprio zelo pela santa Igreja, pela qual Ele derramou Seu Sangue. Ó tu, que foste zeloso pela beleza da Esposa, vela por ela ainda; livra-a dos mercenários que desejam, por vontade dos homens, ocupar o lugar do Esposo. Em nossos dias, a venalidade é menos temida que o compromisso. A simonia tomaria outra forma; não há tanto perigo de suborno como de bajulação, homenagens, acordos e contratos implícitos; tudo isso é tão contrário aos santos Cânones quanto as transações pecuniárias.

E afinal, o mal é menor por tomar forma mais branda, se permite que príncipes voltem a acorrentar a Igreja com as mesmas correntes que lutaste para romper? Não permitas, ó João Gualberto, tal desgraça, que seria prenúncio de terríveis calamidades. Continua a apoiar com teu braço poderoso a Mãe comum dos homens. Salva tua pátria uma segunda vez, mesmo contra sua vontade. Protege, nestes tempos difíceis, a Ordem da qual és glória e pai; dá-lhe força para sobreviver às perseguições e crueldades que sofre daquela mesma Itália que um dia te saudou como libertador. Concede aos cristãos de todas as condições a coragem necessária para o combate ao qual todos somos chamados.

Amém.

[Mencione aqui sua intenção...]

\begin{verse}

\versicle. \quad \textbf{São João Gualberto,}\par
\response. \quad Rogai por nós.

\versicle. \quad \textbf{São João Gualberto,}\par
\response. \quad Rogai por nós.

\versicle. \quad \textbf{São João Gualberto,}\par
\response. \quad Rogai por nós.

\end{verse}

\end{justify}

\vfill

Créditos: Adaptado e traduzido de publicação revisada em inglês.

\end{document}

