% === Template 2: Novena COM orações iguais PARA TODOS OS DIAS ===
\documentclass[18pt]{article}

\usepackage[utf8]{inputenc}
\usepackage[T1]{fontenc}
\usepackage{ragged2e}
\usepackage{caladea}
\usepackage{graphicx}
\usepackage{hyperref}
\usepackage{fancyhdr}

\author{Adaptado de publicação revisada}
\date{} % data fixa ou móvel, se desejar

% Rodapé com imagem e página
\pagestyle{fancy}
\fancyhf{}
\fancyfoot[LO, CE]{%
  \includegraphics[scale=0.2]{assets/cross.png}\quad
  \textit{Novena a \textbf{<Nome do Santo>}}
}
\fancyfoot[R]{\thepage}

\renewcommand{\contentsname}{Sumário}

\begin{document}

\tableofcontents
\thispagestyle{empty}

% --- Vida / Origem da Novena ---
\newpage
\section{Vida / Origem da Novena}
\begin{justify}
  % Aqui você conta a vida ou origem da devoção (trezena, trintena, quaresma, etc.)
  [Escreva aqui uma breve biografia ou origem da novena.]
\end{justify}

% --- Oração Inicial (opcional) ---
\newpage
\section{Oração Inicial}
\begin{justify}
  % Texto da oração inicial, se houver
  [Insira aqui a oração que abre a novena.]
\end{justify}

% --- Orações Diárias Iguais ---
\newpage
\section{Novena a \textbf{<Nome do Santo>}}
\begin{justify}
  % Este texto será repetido em cada um dos nove dias
  [Insira aqui a oração que será rezada em todos os dias da novena.]

  % Se desejar, repita o mesmo texto abaixo em nove seções:
  %
  % \subsection*{Dia 1}
  % [Oração única para todos os dias.]
  %
  % …
  %
  % \subsection*{Dia 9}
  % [Oração única para todos os dias.]
\end{justify}

% --- Oração Final ---
\newpage
\section*{Oração Final}
\begin{justify}
  % Oração de encerramento
  [Insira aqui a oração final, ex.: Pai-Nosso, Ave-Maria e Glória.]
\end{justify}

\end{document}
