% Novena a Santa Maria Madalena
\documentclass[a4paper,12pt]{extarticle}
\usepackage[utf8]{inputenc}
\usepackage[T1]{fontenc}
\usepackage[margin=2.5cm]{geometry}
\usepackage{lmodern}
\usepackage{indentfirst}
\setlength{\parindent}{1.5em}
\usepackage{ragged2e}
\usepackage{graphicx}
\usepackage{hyperref}
\usepackage{fancyhdr}
\usepackage{xcolor}
\usepackage{rotating}
\usepackage{titlesec}

% Centralizar títulos
\titleformat{\section}{\normalfont\centering\bfseries\Large}{\thesection}{1em}{}
\titleformat{\subsection}{\normalfont\centering\bfseries\large}{\thesubsection}{1em}{}
\titleformat{\subsubsection}{\normalfont\centering\bfseries}{\thesubsubsection}{1em}{}


% -------------- Símbolos de Versículo e Resposta --------------
% Definição do símbolo (a “barrinha” inclinada)
\makeatletter
\newcommand{\vers@resp@sym}{%
  \raisebox{0.2ex}{\rotatebox[origin=c]{-20}{$\m@th\rceil$}}%
}
% macro interna que sobrepõe a barrinha e a letra V ou R
\newcommand{\vers@resp}[2]{%
  {\ooalign{%
     \hidewidth\kern#1\vers@resp@sym\hidewidth\cr
     #2\cr
  }}%
}
% comandos públicos \versicle e \response
\DeclareRobustCommand{\versicle}{\vers@resp{-0.1em}{V}. \quad}
\DeclareRobustCommand{\response}{\vers@resp{0pt}{R}. \quad}
\makeatother
% ^------------- Símbolos de Versículo e Resposta -------------^

% Rodapé
\pagestyle{fancy}
\fancyhf{}
\fancyfoot[LO,LE]{\includegraphics[scale=0.2]{assets/cross.png}\quad \textit{Novena a \textbf{Santa Maria Madalena}}}
\fancyfoot[RO,RE]{\thepage}

\renewcommand{\contentsname}{Sumário}

\begin{document}

\begin{center}
  {\LARGE Novena a Santa Maria Madalena}\\[0.5em]
  “Se tem a coragem de imitar Maria Madalena em seus pecados, por que não tem a coragem de imitá-la em seu arrependimento?”\\
  \textit{São Padre Pio a uma penitente}
\end{center}

\tableofcontents
\thispagestyle{empty}

\newpage
\section{Breve Biografia}
Santa Maria Madalena, uma grande pecadora convertida, tornou-se seguidora de Cristo e é, desde os primeiros tempos, o exemplo clássico do pecador arrependido, sendo até uma das Padroeiras dos Dominicanos (Ordem dos Pregadores). É tradicionalmente identificada como a pecadora que ungiu os pés de Jesus Cristo na casa de Simão (Lucas 7:36 e seguintes) e tudo indica que seja Maria, irmã de Marta e de Lázaro.

Segundo os Evangelhos, Jesus expulsou dela sete demónios (Marcos 16:9; Lucas 8:2); acompanhou e serviu o Senhor dos senhores na Galileia (Lucas 8:2), esteve presente na Crucificação (Mateus 27:56; Marcos 15:40; João 19:25), e, com Maria mãe de Tiago e Salomé, descobriu o túmulo vazio e ouviu o anúncio da Ressurreição feito por anjos (Mateus 28:1 e seguintes; Marcos 16:1–8; Lucas 24:1–10).

Foi a primeira pessoa a ver Cristo ressuscitado nesse mesmo dia (Mateus 28:9; Marcos 16:9; João 20:1–18). O seu dia litúrgico celebra-se a 22 de julho, como tal, a Novena inicia-se a 13 de julho.

\section{Perspetiva Mais Detalhada}
As principais fontes sobre Santa Maria Madalena são o Novo Testamento e a Tradição. Nas artes, é frequentemente representada ao lado de Cristo na casa de Simão, aos pés da Cruz, ou sozinha com um vaso de óleo precioso e/ou um livro, símbolos do conhecimento adquirido de Jesus e de Maria. O seu nome completo provém da cidade de onde era originária, Magdala, próxima de Betânia, cuja localização hoje encontra-se em ruínas, com um nome ligeiramente alterado.

Quando Jesus iniciou a sua pregação itinerante, deixou de ter residência fixa, sendo acolhido por amigos ou pessoas hospitaleiras. A casa de Marta, Maria (de Magdala) e do seu irmão Lázaro, em Betânia, a apenas cerca de três quilómetros de Jerusalém, era provavelmente o local onde mais se hospedava. Maria e a sua irmã eram profundamente devotas a Cristo e foi na sua casa que Ele passou os últimos dias antes da Paixão.

No sábado anterior ao Domingo de Ramos, Santa Maria Madalena demonstrou total reverência ao Filho de Deus ao derramar nardo precioso sobre os pés de Jesus – um perfume cujo valor equivaleria a um ano de salário de um trabalhador comum. Este gesto não só expressou o seu arrependimento por uma vida passada de pecado, mas também prenunciou o sacrifício que Cristo faria pela salvação da humanidade poucos dias depois. Antes de ungir os pés de Jesus, Maria seguiu o costume da época para hóspedes honrados e ungiu-lhe primeiro a cabeça. Esta ação excecional foi inspirada pelo Espírito Santo e agradou ao Pai Celeste, tal como testemunhado pelas palavras de Jesus em Mateus 26:10–13.

\section{Frutos da Devoção}
Santa Gertrudes relata que Santa Maria Madalena disse a Santa Matilde:

\begin{quote}
“Aquele que der graças a Deus por todas as lágrimas que derramei sobre os pés de Jesus, o nosso Deus riquíssimo em misericórdia, esse alcançará, por minha intercessão, a remissão de todos os seus pecados antes da morte, bem como um grande aumento no amor a Deus.”
\end{quote}

\section{Oração a Nosso Senhor Jesus Cristo pelas Lágrimas}
Ó dulcíssimo e misericordiosíssimo Jesus, dou-Vos graças pela piedosa obra que a bem-aventurada Maria Madalena realizou em Vós, quando Vos lavou os pés com as suas lágrimas, os enxugou com os cabelos da sua cabeça, os beijou e os ungiu com precioso unguento. Por esse acto de amor, Vos dignastes conceder-lhe tão insigne graça, que infundistes no seu coração e na sua alma um amor tão grande por Vós, que nada mais pôde amar senão a Vós.

Suplico-Vos, por seus méritos e pela sua poderosa intercessão, que Vos digneis conceder-me lágrimas de verdadeira penitência, e derramar no meu coração o Vosso divino amor. \textbf{Amén.}

(Rezar um Pai-Nosso, uma Avé-Maria e um Glória ao Pai.)

\section{Oração Especialíssima a Santa Maria Madalena}
\textit{Composta por Santo Anselmo, Bispo e Doutor da Igreja}

Ó Santa Maria Madalena, vós que, com lágrimas vivas e contritas, vos aproximastes da Fonte da misericórdia, o Senhor Jesus Cristo, e por Ele fostes abundantemente saciada; vós a quem, pela mesma misericórdia, foram perdoados todos os pecados, e cujo coração amargurado foi consolado pela presença do Salvador.

Ó gloriosa penitente, bem sabeis, por experiência própria, como pode uma alma pecadora reconciliar-se com o seu Criador, que conselho há-de receber a alma aflita, e que remédio pode restaurar o enfermo à graça da vida.

Nós, humildes servos de Deus, reconhecemos em vós a grandeza daquele amor que alcança misericórdia, pois a quem muito foi perdoado, é porque muito amou.

Ó bendita entre as mulheres, eu, o mais indigno dos pecadores, não recordo os vossos pecados como acusação, mas antes invoco a inesgotável misericórdia de Deus, pela qual eles foram totalmente apagados.

Nisto repousa a minha esperança, para que não caia no desespero; nisto se acende o meu desejo, para que não pereça.

Vejo-me, miserável e prostrado nas profundezas do vício, esmagado pelo peso das culpas, encerrado nas trevas do pecado. E por isso, vós que estais agora entre os eleitos de Deus, amada do Senhor, intercedei por mim que me encontro ainda em trevas; suplicai por aquele que vos invoca com Fé: alcançai-me luz na escuridão, redenção no pecado, pureza na impureza.

Lembrai-vos com amor do que fostes, de quanto necessitastes da misericórdia divina; e pedi por mim, para que receba o mesmo perdão que vós recebestes. Sede minha advogada, e alcançai-me do Senhor lágrimas de sincero arrependimento, amor que fira o coração, desejo ardente pela pátria celeste, impaciência por este exílio terrestre, compunção verdadeira, e santo temor dos castigos eternos.

Fazei uso da graça que vos foi concedida, e que ainda conservais junto da Fonte de toda a misericórdia.

Atrai-me a Ele, para que, com lágrimas, lave os meus pecados; conduzi-me Àquele que pode saciar a minha alma sedenta; derramai sobre mim as águas vivas que restauram o coração ressequido. Nada vos será recusado por Aquele que tanto ama, e em quem tanto sois amada; Ele, que vive e reina para sempre, é vosso amigo e vosso Senhor.

Quem poderá narrar, ó escolhida de Deus, com que amor Ele tomou a vossa defesa contra os que vos acusavam? Como vos justificou perante o fariseu arrogante; como vos exaltou diante da murmuração da vossa irmã; como louvou o vosso gesto de amor, desprezado por Judas?

E sobretudo, quem poderá exprimir a veemência do vosso amor, quando chorastes no sepulcro, buscando o Senhor? Com que ternura Ele Se vos revelou, depois de Se ocultar para provar a vossa Fé? Como vos chamou pelo nome, com voz tão doce e familiar, dizendo: “Maria!” — e vós O reconhecestes como “Mestre!”

Ó Senhor dulcíssimo, por que perguntas: “Mulher, por que choras?” Não vedes que o seu coração, vida da sua alma, estava ferido pela vossa ausência? Vós, que fostes pregado à Cruz por amor, permitistes que ela sofresse ainda por não encontrar sequer o vosso Corpo, e quisestes que, por entre lágrimas, se manifestasse a profundidade do seu amor.

Mas não pudestes ocultar-Vos por muito tempo daquele coração inflamado por Vós. A força do Vosso amor venceu a dor das suas lágrimas, e manifestastes-Vos a ela com palavras de consolo e certeza: “Eis-Me aqui. Sou Eu a quem procuras.”

Ó que mudança bendita de lágrimas! Se antes brotavam da dor, agora correm da alegria. A alma que dizia: “Levaram o meu Senhor, e não sei onde O puseram,” proclama agora: “Vi o Senhor, e Ele falou comigo.”

E eu, pobre e sem caridade, como ouso falar do Vosso amor, Senhor, e do amor desta vossa fiel discípula? Mas Vós, que sois a própria Verdade, sabeis que o faço por amor do vosso Amor, ó meu dulcíssimo Jesus.

Inflamai o meu coração com aquele amor que ordenais, para que só a Vós ame e me ofereça com espírito contrito, com coração arrependido e humilhado.

Concedei-me, Senhor, neste exílio, o pão das lágrimas e da dor, que desejo mais do que qualquer prazer do mundo.

Escutai-me, por amor do Vosso Amor, pelos méritos da vossa bem-amada Maria Madalena, e da vossa Santíssima Mãe, a Virgem Maria, Rainha dos céus.

Redentor meu, não desprezeis a oração de quem pecou contra Vós, mas fortalecei os esforços do fraco que Vos ama.

Despertai o meu coração da tibieza, Senhor, e, abrasado no Vosso amor, conduzi-me à visão eterna da vossa glória, onde, com o Pai e o Espírito Santo, viveis e reinais, Deus por todos os séculos dos séculos. \textbf{Ámen.}

\section{Ladainha de Santa Maria Madalena}
\begin{description}
  \item \versicle \textbf{Senhor,} 
  \item \response tende piedade de nós.
  \item \versicle \textbf{Jesus Cristo,} 
  \item \response tende piedade de nós.
  \item \versicle \textbf{Senhor,} 
  \item \response tende piedade de nós.
  \item \versicle \textbf{Jesus Cristo,} 
  \item \response ouvi-nos.
  \item \versicle \textbf{Jesus Cristo,} 
  \item \response atendei-nos.
  \item \versicle \textbf{Deus Pai do Céus,} 
  \item \response tende piedade de nós.
  \item \versicle \textbf{Deus Filho,} 
  \item \response Redentor do mundo, tende piedade de nós.
  \item \versicle \textbf{Deus Espírito Santo,} 
  \item \response tende piedade de nós.
  \item \versicle \textbf{Santíssima Trindade,} 
  \item \response tende piedade de nós.
  \item \versicle \textbf{Santa Maria,} 
  \item \response Mãe de Deus, rogai por nós.
  \item \versicle \textbf{Refúgio dos pecadores,} 
  \item \response rogai por nós.
  \item \versicle \textbf{Rainha concebida sem pecado,} 
  \item \response rogai por nós.
  \item \versicle \textbf{Santa Maria Madalena,} 
  \item \response rogai por nós.
  \item \versicle \textbf{Irmã de Marta e Lázaro,} 
  \item \response rogai por nós.
  \item \versicle \textbf{Vós que entrastes na casa do fariseu para ungir os pés de Jesus,} 
  \item \response rogai por nós.
  \item \versicle \textbf{Que lavastes os Seus pés com as vossas lágrimas,} 
  \item \response rogai por nós.
  \item \versicle \textbf{Que enxugastes com os vossos cabelos,} 
  \item \response rogai por nós.
  \item \versicle \textbf{Que cobristes de beijos,} 
  \item \response rogai por nós.
  \item \versicle \textbf{Que fostes defendida por Jesus diante do fariseu,} 
  \item \response rogai por nós.
  \item \versicle \textbf{Que d’Ele recebestes o perdão dos pecados,} 
  \item \response rogai por nós.
  \item \versicle \textbf{Que das trevas fostes conduzida à luz,} 
  \item \response rogai por nós.
  \item \versicle \textbf{Apóstola dos Apóstolos,} 
  \item \response rogai por nós.
\end{description}

\bigskip
\textbf{Oração Final}
\begin{quote}
Ó meu Deus, seja o Vosso amor o único princípio da minha penitência. Que a dor do meu coração não tenha outra causa senão a de Vos ter ofendido, a Vós, Bem supremo, meu princípio e fim último. Fazei que as minhas lágrimas brotem de um coração ferido pelo vosso amor. Arrependo-me sinceramente do passado, e proponho firmemente nunca mais Vos ofender. Por intercessão de Santa Maria Madalena, concedei-me a graça da verdadeira conversão, da pureza de coração e do ardente amor com que ela Vos amou. Por Cristo Nosso Senhor. \textbf{Ámen.}
\end{quote}

\end{document}
