\documentclass[a4paper,12pt]{extarticle} \usepackage[utf8]{inputenc}
\usepackage[T1]{fontenc}
\usepackage[margin=2.5cm]{geometry}

% Fonte Caladea se existir, senão lmodern
\IfFileExists{caladea.sty}{
  \usepackage{caladea}
}{
  \usepackage{lmodern} }
\usepackage{ragged2e}
\usepackage{graphicx}
\usepackage{hyperref}
\usepackage{fancyhdr}
\usepackage{xcolor}
\usepackage{rotating}
\usepackage{titlesec}
\usepackage{dirtytalk}
\usepackage[portuguese]{babel}
\usepackage{indentfirst} % Indenta o primeiro parágrafo após seções

% Ajuste do recuo de parágrafo
\setlength{\parindent}{1.5em}

% Centralizar títulos
\titleformat{\section}
  {\normalfont\centering\bfseries\Large}{\thesection}{1em}{}

\titleformat{\subsection}
  {\normalfont\centering\bfseries\large}{\thesubsection}{1em}{}

\titleformat{\subsubsection}
  {\normalfont\centering\bfseries}{\thesubsubsection}{1em}{}

% -------------- Símbolos de Versículo e Resposta --------------
% Definição do símbolo (a “barrinha” inclinada)
\makeatletter
\newcommand{\vers@resp@sym}{%
  \raisebox{0.2ex}{\rotatebox[origin=c]{-20}{$\m@th\rceil$}}%
}
% macro interna que sobrepõe a barrinha e a letra V ou R
\newcommand{\vers@resp}[2]{%
  {\ooalign{%
     \hidewidth\kern#1\vers@resp@sym\hidewidth\cr
     #2\cr
  }}%
}
% comandos públicos \versicle e \response 
\DeclareRobustCommand{\versicle}{\vers@resp{-0.1em}{V}.  } 
\DeclareRobustCommand{\response}{\vers@resp{0pt}{R}.  }
\makeatother
% ^------------- Símbolos de Versículo e Resposta -------------^

% Rodapé com imagem e página
\pagestyle{fancy}
% ---- Cabeçalho ------------
\fancyhf[C]{}
% ----- Rodapé --------------
\fancyfoot[LO,LE]{%
  \includegraphics[scale=0.2]{assets/cross.png}\quad
  \textit{Novena aos \textbf{14 Santos Auxiliares}}
}
\fancyfoot[RO,RE]{\thepage}

\begin{document}

\begin{center}
  \textbf{\LARGE Novena aos 14 Santos Auxiliares}\\[0.5em]
  \say{Durante o período devastador da Peste Negra, que assolou a Europa entre 1346 e 1349, muitos santos foram invocados pelos fiéis contra a peste e a morte súbita. Entre eles estavam aqueles que, um século depois, ficariam conhecidos como os Quatorze Santos Auxiliares.}\\
\end{center}

\tableofcontents
\thispagestyle{empty}

% --- Vida / Origem da Novena ---
\newpage

\section{Origem da Devoção}

Os 14 santos auxiliares são:

São Jorge, São Brás, Santo Erasmo, São Pantaleão, São Vito, São Cristóvão, São Denis, São Ciríaco, Santo Acácio, Santo Eustáquio, Santo Egídio, Santa Margarida, Santa Catarina Alexandria, Santa Bárbara.

Eles eram invocados em conjunto por causa da peste negra que devastou a Europa de 1346 a 1349.

Os piedosos invocaram os santos suplicando pela sua intervenção e vários se curaram ou não contraíram a doença.


\subsection{São Jorge -  Valente mártir de cristo} \label{são-jorge}

Nasceu na Capadócia(região que atualmente pertence à Turquia)

Após a morte de seu pai, Jorge mudou-se para a Palestina com sua mãe.

Como era muito dedicado e habilidoso foi promovido a capitão do exército romano e recebeu o título de conde

Com 23 anos passou a residir na corte imperial em Roma, exercendo altas funções.

O imperador Diocleciano tinha planos para matar todos os cristãos.

No dia marcado para o Senado confirmar este decreto imperial, Jorge levantou-se no meio da reunião para defender o cristianismo e afirmou que os ídolos adorados nos templos pagãos eram falsos deuses.

Todos ficaram espantados ao ouvirem estas palavras de um membro da suprema corte romana, que estava defendendo a fé em Jesus Cristo como Senhor e Salvador dos homens.

Um cônsul perguntou da onde vinha tamanha ousadia

Jorge prontamente respondeu-lhe que era por causa da VERDADE.

O cônsul, não satisfeito, quis saber: “o que é a verdade ?”

Jorge respondeu: “A verdade é meu Senhor Jesus Cristo, a quem vós perseguis, e eu sou servo de meu redentor Jesus Cristo, e nele confiado me pus no meio de vós para dar testemunho da verdade.”

O Imperador tentou fazer São Jorge desistir de sua fé em Jesus Cristo por meio de diversas torturas

E, após cada tortura, era levado perante o imperador, que lhe perguntava se agora renegaria a Jesus Cristo para adorar os ídolos.

Jorge sempre respondia: “Não, imperador! Eu sou servo de um Deus vivo! Somente a Ele eu temerei e adorarei”.

Com tamanho testemunho de São Jorge, muitos que presenciaram sua fé passaram a crer em Jesus Cristo, tornando-se cristãos


 
\subsection{São Brás – Zeloso bispo e benfeitor dos pobres}

São Brás nasceu na cidade de Sebaste, Armênia, no final do século III.

Era um ótimo médico, cuidava da pessoa em sua totalidade, cada vez se entregava mais a Cristo

Cada vez passou a buscar mais a Jesus e evangelizar aqueles que tinha contato

Seu amor crescente fez ser totalmente de Cristo.

Tornou-se Bispo

Viveu no tempo do Imperador Licínio, era em um tempo de muita perseguição ao cristianismo


São Brás foi preso e sofreu muitas chantagens para que abandonasse à fé católica.

Em 316, foi degolado.

Ao se dirigir para o martírio, uma mãe apresentou-lhe uma criança de colo que estava morrendo engasgada por causa de uma espinha de peixe na garganta.

Ele parou, olhou para o céu, rezou e Nosso Senhor curou aquela criança.


\subsection{Santo Erasmo – Poderoso protetor dos oprimidos}

Santo Erasmo pertencia ao clero de Antioquia.

Durante a perseguição do Imperador Diocleciano ficou em uma caverna durante 7 anos

Foi capturado e longamente torturado

Foi levado para ser julgado pelo imperador que tentou de todas as formas fazer com que renegasse a fé em Cristo.

Santo Erasmo se manteve firme e por isso novamente voltou para a prisão.

Nesta prisão foi milagrosamente libertado por um anjo que o levou para a Dalmácia

Na Dalmácia, durante 7 anos fez milhares de conversões

Na época do imperador Maximiano, novamente foi preso e no tribunal, onde destruiu um ídolo e declarou -se cristão.

Esta atitude de Erasmo fez milhares de pagãos se converterem

O Imperador ficou furioso e intensificou a sua perseguição

Em uma outra ocasião foi horrivelmente torturado

 

Sofreu muito durante suas torturas, seu ventre fora cortado e aos poucos os seus intestinos eram retirados. Por causa deste suplício Santo Erasmo se tornou para os fiéis o protetor das enfermidades do ventre, dos intestinos e das dores do parto.

Mas depois foi libertado por São Miguel Arcanjo que o conduziu para a costa do sul da Itália.

Ali se tornou o Bispo de Fornia, mas por um breve período.

Morreu pouco depois por casa das feridas de seus dois suplícios, por este motivo recebeu o título de Mártir.
 

\subsection{São Pantaleão – Milagroso exemplar da caridade}

Viveu no século III- IV, em que ocorria uma intensa perseguição aos católicos

Pantaleão era filho de Eustóquio, gentio e de Êubola, cristã.

Sua mãe ensinou-lhe a fé cristã.

Após a morte de sua mãe, Pantaleão foi encaminhado pelo pai aos estudos de retórica, filosofia e medicina.

Durante este período, tornou-se amigo do padre Hermolau, exemplo de virtude, que o convenceu que Nosso Senhor Jesus Cristo é o autor da vida e o senhor da verdadeira saúde.

Um dia encontrou uma criança morta por uma cobra e disse:

“Agora verei se é verdade o que Hermolau me diz”.

E disse para o menino: “Em nome de Jesus Cristo, levanta-te; e tu, animal peçonhento, sofre o mal que fizeste”.

Levantou-se a criança e a cobra ficou morta

Pantaleão converteu-se e recebeu logo o Santo Batismo.

Foi convocado pelo imperador Maximiano como seu médico pessoal.

Por causa das milagrosas curas que realizava em nome de Jesus Cristo causaram a inveja de outros médicos

Esses médicos invejosos entregaram Pantaleão para o imperador que, por sua vez, o mandou ser amarrado a uma árvore e degolado.


\subsection{São Vito – Especial protetor da castidade}

Nasceu na Itália, no século III, durante a perseguição do Imperador Dioclesiano

Era filho de Halaz, rico e nobre senhor romano, que tinha grandes dignidades, honras e nobreza.

É provável que sua mãe também pertencesse à nobreza

Halaz obedecia as ordens do Imperador e perseguia os cristãos (só em um mês, Diocleciano sacrificou dezessete mil mártires.)

Sua mãe morreu quando Vito ainda era muito pequeno e foi entregue para uma ama de leite chamada Crescencia, uma mulher virtuosa e cristã

Quando Vito cresceu um pouco mais foi entregue aos cuidados de Modesto, esposo de Crescencia, que também era cristão.

Tanto Crescencia, quanto Modesto, nunca revelaram que eram cristãos, pois sabiam que o senhor Halaz era um perseguidor dos cristãos.

Assim Vito foi educado na fé católica e batizado

Seu pai quando ficou sabendo que seu filho Vito era batizado, tentou fazer com que ele abandonasse a fé

Como Vito não negou Jesus Cristo, castigou o próprio filho, levou-o para a prisão, onde foi maltratado durante vários dias

Com a ajuda de um Anjo, Vito escapou da prisão

Vito fugiu com Crescencia e Modesto para Lucânia, em Napoles

Depois de um certo tempo foram reconhecidos e precisaram fugir

Vito, desde os 7 anos havia manifestado dons especiais e durante esta época, ele ressuscitou em nome de Jesus, um garoto que tinha sido morto por cães raivosos

A perseguição só teve uma pausa, quando o filho do Imperador, que tinha epilepsia, ficou muito doente.

Por isso, Diocleciano que ficou sabendo dos dons de Vito o chamou e pediu que curasse o filho

Vito rezou e conseguiu a cura do filho do Imperador

Diocleciano não lhe teve gratidão e mandou prender Vito

Como Vito não negou sua fé, foi condenado a morte com 15 anos de idade

Modesto e Crescencia foram levados para o Circo, sofreram muitas torturas e jogados para cães raivosos, mas por milagre, os cães ao invés de atacarem, deitavam-se aos seus pés. Diocleciano então mandou que eles fossem levados para um caldeirão de óleo quente, onde morreriam lentamente

São Venceslau tinha muita devoção por São Vito e até construiu uma belíssima catedral, onde estão suas relíquias


\subsection{São Cristovão – Poderoso intercessor nos perigos}

Seu nome era Ofero Relicto, depois mudou para Chistophoros (Cristóvão) que em grego significa “aquele que carrega Cristo”

Cristóvão foi um homem de estatura alta e de grande força e decidiu procurar homens mais fortes do que ele para servi-los.

Encontrou um rei poderoso e forte, mas este tinha medo do diabo.

Então decidiu servir a outro senhor que dizia ser o próprio satanás.

Mas observou que Satanás tinha medo da cruz e de tudo o que dizia respeito a Jesus Cristo.

Por isso resolveu servir ao Senhor Jesus.

Um monge catequizou-o na fé cristã e foi batizado.

Decidiu morar à margem de um rio muito perigoso para fazer a travessia de pessoas, as quais levava em seus ombros.

Certo dia transportou um menino que ficava cada vez mais pesado à medida que o levava.

Ele sentia estar levando todo o peso do mundo em seus ombros.

Cristóvão estranhou aquele peso: «Parece que estou a carregar o mundo inteiro!»

Sorridente, a criança explicou: «Estás a carregar muito mais do que o mundo inteiro. Carregaste o Senhor do Mundo.»

No final da travessia, o menino Jesus, confirmou que sua missão era a humilde ajuda na travessia das pessoas

Por isto mudou seu nome para "Cristóvão", que significa: "Aquele que carrega Cristo".

A partir de então, São Cristóvão via Jesus Cristo em todas as pessoas.

Seu amor e sua vida converteu muitos na região

Sofreu o martírio
 

\subsection{São Denis – Brilhante espelho de fé e confiança}

São Dênis nasceu na França, era um jovem missionário que foi enviado pelo Papa Fabiano para evangelizar a região da França

Criou uma comunidade católica em Lutécia (atual París), onde foi eleito o primeiro bispo

Depois de alguns anos sofreu o martírio da decapitação no lugar conhecido como Montmartre, isto é, "Colina do Mártir".

Ao lado do bispo Dênis, o sacerdote Rústico e o diácono Eleutério, seus companheiros, também testemunharam sua fé cristã.

Sobre seu túmulo foi construída uma Basílica e no ano de 630 foi construída uma abadia pelo rei Dagoberto

É aclamado como um mártir cefalóforo, ou seja, carregador de cabeça, isto porque São Dênis, depois de ser martirizado carregou a própria cabeça decepada até o local onde deveria ser enterrado.

\subsection{São Ciríaco – Terror dos infernos}

Seu nome era Judas e ajudou Santa Helena no trabalho de encontrar a Cruz de Cristo na cidade de Jerusalém

Judas se converteu, tornou-se padre e mudou seu nome para Ciríaco

Foi eleito bispo, trabalhou arduamente pela evangelização

Foi martirizado, junto com sua mãe, chamada Ana, durante a perseguição de Juliano, o Apóstata.

\subsection{Santo Acácio – Eficaz advogado na hora da morte}

CONTRA AS DORES E DEMAIS MALES DA CABEÇA

Acácio era um centurião da Capadócia(atual Turquia), do exército romano da cidade de Trácia.

Como era cristão sofreu muitas torturas, flagelos e tormentos até ser martirizado por decapitação

O imperador Constantino construiu um santuário em sua honra em Karía de Constantinopla. Nesta cidade Santo Acácio se tornou o padroeiro.

Acácio reza a Deus  pouco antes de sua morte para que quem venerar a sua memória se beneficiaria com a saúde da mente e do corpo, foi por esta razão que Santo Acácio foi incluído entre os populares Quatorze Santos Auxiliares.


\subsection{Santo Eustáquio – Exemplo de paciência  na adversidade}

Seu nome era Placidus e foi um general romano.

Esta mudança de nome aconteceu quando Placidus encontrou um veado que trazia entre os chifres um crucifixo resplandecente.

Depois deste encontro, Placidus tornou-se cristão e batizou-se com o nome de Eustáquio, junto dele estavam também a mulher e seus dois filhos.

O Imperador Adriano prometeu o perdão a toda a família, se renegassem a fé e fizessem sacrifícios aos ídolos.

A família se recusa e sofre o martírio


\subsection{Santo Egídio – Menosprezo do mundo terrestre}

Egídio (Gilles) Nasceu em Provence, França no século VIII e foi Abade do Monastério Beneditino de Rhone River( hoje é o Monastério de Saint-Gilles)

Giles tornou-se um eremita na nascente do Rio Rhône.

Sua única companhia era uma corsa que procurou refúgio das caçadas organizadas pelo Rei dos Goths.

Certo dia um dos melhores caçadores do rei atirou uma flecha certeira em uma corsa e em vez dela estar morta, encontraram São Giles segurando a flecha e a corsa.

São Giles teria desviado a flecha para que a flecha acertasse a sua mão e não a corça.

O rei ficou impressionado e construiu um pequeno monastério para Giles.

Mais tarde Calos Magno o chamou para ser seu confessor.

Ele foi um dos santos mais populares da Idade Média, vários milagres ocorreram pela sua intercessão

\subsection{Santa Margarida -  Valente campeã da fé}

Margarida era filha de um sacerdote pagão de Antioquia, chamado Edesio.

Converteu-se ao cristianismo e foi expulsa da casa do pai.

Procurou e encontrou a mulher que a amamentara quando era menor, passaram a viver juntas novamente e tornou-se pastora.

Quando tinha quinze anos foi vista pelo prefeito Olíbrio, que, encantado com tanta beleza, queria se casar com ela.

Margarida disse que era cristã e recusou a proposta

O prfeito Olíbrio ficou com muita raiva e mandou vários soldados prendê-la.

O prefeito ordenou que a chicoteassem e lhe rasgassem os flancos com as unhas de ferro.

Depois de todo sofrimento foi levada para a prisão

Sem soltar uma queixa, Margarida encostou-se a um canto da prisão.

O diabo sob forma de um feio dragão, apareceu-lhe para pervertê-la.

Santa Margarida fez o  sinal da cruz e o afastou

Depois o demônio apareceu sob a forma de um homem peludo, tentou fazer de tudo para desviá-la, mas não conseguia

Apareceu uma luz divina na escuridão da masmorra, que expulsou o demônio e encheu Margarida de alegria

Sempre carregava uma cruz e uma pomba pousou nela, o que lhe deu mais força

No dia seguinte, colocaram tochas ardentes em seu corpo, mas não sentiu nada

Foi atirada em uma caldeira de azeite fervente, mas saiu sem danos

A multidão ficou impressionada, Santa Margarida começou exortar e converteu muitas almas

Os convertidos e Santa Margarida depois foram decapitados


“Quem de ti fizer memória
Força terá, e vitória.
Onde houver relíquia tua,
Ou quem tua Vida folhear,
Deus jamais há de olvidar”


Santa Margarida se manifestou a Santa Joana D'Arc.
 
\subsection{Santa Catarina Alexandria – Vitoriosa defensora da fé}

Santa Catarina de Alexandria nasceu de família nobre e estudou muito.

O imperador romano Maximus fazia uma violenta perseguição aos cristãos e chamou Catarina com seus 18 por ser católica.

O Imperador ficou chocado com as respostas de Santa Catarina e chamou vários sábios para fazer com que Catarina abandonasse sua fé.

Os mais sábios tentavam usar todos os argumentos, mas ela sempre conseguia vencer.

Conseguiu fazer até com que eles se convertessem, que foram condenados a morte

 

Furioso por estar sendo derrotado, o Imperador Maximus prende Catarina.

A imperatriz ficou curiosa para conhecer a jovem que desafiava seu marido e vai acompanhada de Porfírio, chefe das tropas do imperador, até a prisão onde estava Catarina, que também consegue converte-los; todos eles são martirizados.

Catarina foi então condenada à morte na roda de tortura, mas apenas ela encostou na roda para que ela se partisse e matasse vários pagãos que estavam assistindo.

O imperador ficou muito nervoso e ordenou que ela fosse decapitada.

Depois de sua morte, anjos desceram dos céus e levaram seu corpo para o Monte Sinai, onde depois se construiu uma igreja e um mosteiro em sua honra.

\subsection{Santa Bárbara – Poderosa protetora dos moribundos}

Santa Bárbara nasceu na cidade de Nicomédia (na região da Bitínia, Turquia) no final do século III

Era a filha única do rico e nobre Dióscoro

Por ser filha única e com receio de deixar a filha no meio da sociedade corrupta daquele tempo, Dióscoro decidiu fechá-la numa torre.

Santa Bárbara que ficava sozinha na torre e tinha como quinta a mata, questionava-se se de fato, tudo aquilo era criação dos ídolos

Ela era muito bela e tinha muitos pretendentes para casamento, mas Bárbara não aceitava nenhum.

Dióscoro acreditava que as “desfeitas” da filha ocorriam por que ela ficou muitos anos trancada na torre

Então, ele permitiu que ela fosse conhecer a cidade

Durante essa visita ela teve contato com cristãos, que lhe contaram sobre Jesus e sobre a Santíssima Trindade.

Um padre lhe deu o Batismo.

Um tempo depois, seu pai decidiu construir uma casa de banho para Bárbara, que possuía duas janelas.

Alguns dias depois, seu pai precisou fazer uma longa viagem.

Enquanto Dióscoro viajava, sua filha ordenou a construção de uma terceira janela na torre

Além disso, ela esculpira uma cruz sobre a fonte.

Quando seu pai voltou da viagem, reparou que a torre onde tinha trancado a filha tinha agora três janelas em vez das duas que ele mandara abrir.

Ao perguntar à filha o porquê das três janelas, ela explicou-lhe que isso era o símbolo da sua nova Fé e representava a Santíssima Trindade

Este fato deixou o seu pai furioso, porque ela se recusava a seguir a fé dos Deuses do Olimpo.

O pai denunciou sua filha ao Prefeito Martiniano.

O prefeito a mandou torturar para que mudasse de ideia, mas isto não aconteceu.

Assim, o prefeito Marcius condenou-a à morte por degolação.

Bárbara teve os seios cortados

Depois foi conduzida para fora da cidade onde o seu próprio pai a executou, degolando-a.

Quando a cabeça de Bárbara rolou pelo chão, um imenso trovão ribombou pelos ares fazendo tremer os céus.

Um relâmpago flamejou pelos ares e atravessando o céu matou seu pai Dióscoro.


\newpage
\section{Orações}

\subsection{Oração Inicial} \label{oracao-inicial}

\textbf{Em nome do Pai, do Filho e do Espírito Santo. Amém.}

Querido Senhor, agradecemos-Te por nos dar os 14 Santos Auxiliares como exemplos de santidade. Ajuda-nos a imitar a sua virtude heróica nas nossas próprias vidas.

% --- Orações Diárias ---

\subsection{Novena}

\subsubsection{Primeiro Dia}

\noindent


\textbf{Começar com a \nameref{oracao-inicial}.}

Santos Auxiliares, a maioria de vocês passou por muito sofrimento durante sua vida na Terra, mas manteve firme sua fé e seu amor a Deus. Vocês são invocados como padroeiros contra as doenças, e sua intercessão tem ajudado muitas pessoas que sofrem de enfermidades físicas.

Nós vos imploramos que levem todas as nossas petições diante do trono de Deus e, hoje, pedimos especialmente que rezem para que todos aqueles que sofrem de doenças possam encontrar conforto e alívio.

Rezem por nós, para que possamos crescer no amor a Deus a cada dia de nossas vidas. Rezem para que possamos servir a Deus dignamente em todas as circunstâncias que enfrentarmos.


\textbf{Finalizar com a \nameref{ladainha}.}


\subsubsection{Segundo Dia}

\textbf{Começar com a \nameref{oracao-inicial}.}


Santos Auxiliares, a vossa intercessão é conhecida por ser muito poderosa em ajudar aqueles que sofrem de doenças. Vós próprios suportastes muitos sofrimentos físicos. Muitos de vós fostes torturados e mortos pela vossa fé. Sabem como o sofrimento físico pode ser difícil.

Nós vos imploramos que levem todas as nossas petições diante do trono de Deus e, hoje, pedimos especialmente que rezem para que todos aqueles que cuidam dos doentes sejam compassivos e possam trazer alívio aos seus pacientes.

Rezem por nós, para que possamos oferecer tudo o que fazemos a Deus. Rezem para que possamos crescer em santidade e virtude em todas as oportunidades.


\textbf{Finalizar com a \nameref{ladainha}.}


\subsubsection{Terceiro Dia}

\textbf{Começar com a \nameref{oracao-inicial}.}


Santos Auxiliares, vocês demonstraram grande fé no poder de Deus ao suportar torturas e viver com virtude heróica. Vocês sabem que Deus tem poder até mesmo sobre a natureza. Por meio de sua intercessão, vocês podem ajudar todos aqueles que buscam um milagre para serem curados de doenças.

Nós vos imploramos que levem todas as nossas petições diante do trono de Deus e, hoje, pedimos especialmente que rezem por todos aqueles que buscam uma cura milagrosa para suas doenças.

Rezem por nós, para que possamos crescer em nossa fé em Deus a cada dia. Rezem para que possamos sempre confiar em Deus, independentemente das dificuldades que possamos enfrentar.


\textbf{Finalizar com a \nameref{ladainha}.}


\subsubsection{Quarto Dia}

\textbf{Começar com a \nameref{oracao-inicial}.}

Santos Auxiliares, a vossa intercessão é frequentemente invocada por aqueles que sofrem de doenças. Como sofreram tanto pela vossa fé durante a vossa vida na Terra, sabem como os sofrimentos físicos podem ser difíceis. A vossa intercessão pode ajudar todos aqueles que se sentem desanimados no meio do seu sofrimento.

Nós vos imploramos que levem todas as nossas petições diante do trono de Deus e, hoje, pedimos especialmente que rezem por todos aqueles que estão ficando desanimados devido ao sofrimento prolongado.

Rezem por nós, para que possamos sempre perseverar em todas as provações que enfrentarmos. Rezem para que possamos nos tornar verdadeiramente santos, não importa o custo.

\textbf{Finalizar com a \nameref{ladainha}.}


\subsubsection{Quinto Dia}

\textbf{Começar com a \nameref{oracao-inicial}.}

Santos Auxiliares, a vossa intercessão é frequentemente invocada por aqueles que sofrem de doenças. Como sofreram tanto pela vossa fé durante a vossa vida na Terra, sabem como os sofrimentos físicos podem ser difíceis. A vossa intercessão pode ajudar todos aqueles que se sentem desanimados no meio do seu sofrimento.

Nós vos imploramos que levem todas as nossas petições diante do trono de Deus e, hoje, pedimos especialmente que rezem por todos aqueles que estão ficando desanimados devido ao sofrimento prolongado.

Rezem por nós, para que possamos sempre perseverar em todas as provações que enfrentarmos. Rezem para que possamos nos tornar verdadeiramente santos, não importa o custo.

\textbf{Finalizar com a \nameref{ladainha}.}


\subsubsection{Sexto Dia}

\textbf{Começar com a \nameref{oracao-inicial}.}


Santos Auxiliares, vocês viveram o seu grande amor por Deus ao longo das vossas vidas. Muitos de vocês sofreram muito por causa da vossa fé, mas mantiveram-se firmes e recusaram-se a renunciar à vossa fé. Viver e morrer como vocês viveram e morreram exigiu uma grande confiança em Deus.

Nós vos pedimos que levem todas as nossas petições diante do trono de Deus e, hoje, pedimos especialmente que rezem para que possamos crescer na confiança Nele.

Rezem por nós, para que possamos sempre crescer em santidade em todas as oportunidades de nossas vidas. Rezem para que possamos nos esforçar para imitar vossa virtude heróica em todas as oportunidades.


\textbf{Finalizar com a \nameref{ladainha}.}


\subsubsection{Sétimo Dia}

\textbf{Começar com a \nameref{oracao-inicial}.}



Santos Auxiliares, muitos de vocês enfrentaram desafios difíceis para viver a sua fé. Em muitos casos, sabiam que viver a sua fé seria difícil e poderia até levar à morte. Apesar da possibilidade de grande sofrimento, dedicaram-se corajosamente a servir a Deus.

Nós vos imploramos que levem todas as nossas petições diante do trono de Deus e, hoje, pedimos especialmente que rezem para que possamos crescer em coragem.

Rezem por nós, para que nunca permitamos que as dificuldades que possamos enfrentar nos impeçam de servir a Deus. Rezem para que possamos manter nossa fé com devoção, não importa o custo. 


\textbf{Finalizar com a \nameref{ladainha}.}


\subsubsection{Oitavo Dia}

\textbf{Começar com a \nameref{oracao-inicial}.}


Santos Auxiliares, vocês suportaram muito sofrimento durante suas vidas na Terra. Vocês são invocados como padroeiros daqueles que sofrem de doenças. Sua profunda santidade não teria sido possível sem a virtude da perseverança diante de seus sofrimentos. Sua intercessão pode nos ajudar a crescer na perseverança de que precisamos para alcançar a santidade.

Nós vos imploramos que levem todas as nossas petições diante do trono de Deus e, hoje, pedimos especialmente que rezem para que possamos crescer em perseverança.

Rezem por nós, para que possamos fazer tudo o que estiver ao nosso alcance para nos dedicarmos mais plenamente a Deus. Rezem para que possamos fazer do crescimento em santidade nossa principal prioridade na vida.

\textbf{Finalizar com a \nameref{ladainha}.}


\subsubsection{Nono Dia}

\textbf{Começar com a \nameref{oracao-inicial}.}

Santos Auxiliares, muitos de vocês enfrentaram desafios difíceis para viver a sua fé. Em muitos casos, sabiam que viver a sua fé seria difícil e poderia até levar à morte. Apesar da possibilidade de grande sofrimento, dedicaram-se corajosamente a servir a Deus.

Nós vos imploramos que levem todas as nossas petições diante do trono de Deus e, hoje, pedimos especialmente que rezem para que possamos crescer em coragem.

Rezem por nós, para que nunca permitamos que as dificuldades que possamos enfrentar nos impeçam de servir a Deus. Rezem para que possamos manter nossa fé com devoção, não importa o custo. 

\textbf{Finalizar com a \nameref{ladainha}.}

% --- Oração Final ---
\newpage
\subsection{Ladainha aos 14 Santos Auxiliares} \label{ladainha}

Senhor, \textbf{ tende piedade de nós}
                                      
Cristo, \textbf{ tende piedade de nós}
                                      
Senhor, \textbf{ tende piedade de nós}

 

Cristo, \textbf{ouvi-nos}

Cristo, \textbf{atendei-nos}

 

Deus Pai do Céu, \textbf{ tende piedade de nós.}

Deus Filho, \textbf{ Redentor do Mundo, tende piedade de nós.}

Deus Espírito Santo, \textbf{ tende piedade de nós..}

Santíssima Trindade, \textbf{ Um só Deus, tende piedade de nós.}

 

Santa Maria, Rainha dos Mártires, \textbf{rogai por nós.}

São José, Auxiliar em todas as necessidades, \textbf{rogai por nós.}

Catorze Santos Auxiliares, \textbf{rogai por nós.}

 

São Jorge, valente mártir de Cristo, \textbf{rogai por nós.}

 

São Brás, zeloso bispo e benfeitor dos pobres, \textbf{rogai por nós.}

 

Santo Erasmo, poderoso protetor dos oprimidos, \textbf{rogai por nós.}

 

São Pantaleão, miraculoso exemplo de caridade, \textbf{rogai por nós.}

 

São Vito, especial protetor da castidade, \textbf{rogai por nós.}

 

São Cristóvão, poderoso intercessor nos perigos, \textbf{rogai por nós.}

 

São Denis, brilhante espelho de fé e confiança, \textbf{rogai por nós.}

 

São Círiaco, terror do inferno, \textbf{rogai por nós.}

 

Santo Acácio, atencioso advogado na morte, \textbf{rogai por nós.}

 

Santo Eustáquio, exemplo de paciência na adversidade, \textbf{rogai por nós.}

 

São Egídio, que desprezou o mundo, \textbf{rogai por nós.}

 

Santa Margarida, valente campeã da Fé, \textbf{rogai por nós.}

 

Santa Catarina, vitoriosa defensora da Fé e da pureza, rogai por nós.

 

Santa Bárbara, poderosa protetora dos moribundos, \textbf{rogai por nós.}

 

Todos vós Santos Auxiliares, \textbf{rogai por nós.}

Todos os Santos de Deus, \textbf{rogai por nós.}

 

Nas tentações contra a fé, \textbf{rogai por nós.}

 

Nas adversidades e tribulações, \textbf{rogai por nós.}

 

Nas ansiedades e desejos, \textbf{rogai por nós.}

 

Em todo combate \textbf{rogai por nós.}

 

Em toda tentação, \textbf{rogai por nós.}

 

Na enfermidade, \textbf{rogai por nós.}

 

Em todas as necessidades, \textbf{rogai por nós.}

 

No temor e no terror, \textbf{rogai por nós.}

 

Nos perigos à salvação, \textbf{rogai por nós.}

 

Nos perigos à honra, \textbf{rogai por nós.}

 

Nos perigos à reputação, \textbf{rogai por nós.}

 

Nos perigos à propriedade, \textbf{rogai por nós.}

 

Nos perigos por fogo e água, \textbf{rogai por nós.}

 

Sede-se nos propício,\textbf{ perdoai-nos Senhor}

 

Sede-se nos propício, \textbf{escutai-nos, Senhor,}

 

De todo o pecado, \textbf{livrai-nos, Senhor}

 

De vossa Ira, \textbf{livrai-nos, Senhor}

 

Do flagelo do terremoto, \textbf{livrai-nos, Senhor}

 

Da praga, da fome e da guerra, \textbf{livrai-nos, Senhor}

 

Dos raios e tempestades, \textbf{livrai-nos, Senhor}

 

Da morte repentina e imprevista, \textbf{livrai-nos, Senhor}

 

Da condenação eterna, \textbf{livrai-nos, Senhor}

 

Pelo mistério de Vossa Santa Encarnação, \textbf{livrai-nos, Senhor}

 

Pelo Vosso nascimento e Vossa vida, \textbf{livrai-nos, Senhor}

 

Pela Vossa Cruz e Paixão, \textbf{livrai-nos, Senhor}

 

Pela Vossa morte e sepultura, \textbf{livrai-nos, Senhor}

 

Pelos méritos de Vossa Santíssima Mãe Maria, \textbf{livrai-nos, Senhor}

 

Pelos méritos dos Catorze Santos Auxiliares,\textbf{ livrai-nos, Senhor}.

 

No dia do Julgamento,\textbf{ livrai-nos, Senhor}.

 

Pecadores que somos, nós \textbf{Vos rogamos, ouvi-nos.}

 

Para que nos perdoeis, nós \textbf{Vos rogamos, ouvi-nos.}

 

Para que nos favoreçais, nós \textbf{Vos rogamos, ouvi-nos.}

 

Para que Vos digneis conduzir-nos a uma verdadeira penitência, nós \textbf{Vos rogamos, ouvi-nos.}

 

Para que Vos dignes dar e preservar os frutos da terra, nós \textbf{Vos rogamos, ouvi-nos.}

 

Para que Vos digneis governar e propagar a vossa Santa Igreja, nós \textbf{Vos rogamos, ouvi-nos.}

 

Para que Vos digneis conceder a paz e concórdia entre as nações, nós \textbf{Vos rogamos, ouvi-nos.}

 

Para que Vos digneis dar a todos os fiéis defuntos o descanso eterno, nós \textbf{Vos rogamos, ouvi-nos.}

 

Para que Vos digneis vir em nosso auxílio por intercessão dos Santos Auxiliares, nós \textbf{Vos rogamos, ouvi-nos.}

 

Para que Vos digneis, por intercessão de São Jorge, preservar-nos na Fé, nós \textbf{Vos rogamos, ouvi-nos.}

Para que Vos digneis, por intercessão de São Brás, confirmar-nos na Esperança, nós \textbf{Vos rogamos, ouvi-nos.}

 

Para que Vos digneis, por intercessão de Santo Erasmo, inflamar-nos em Vosso Santo amor, nós \textbf{Vos rogamos, ouvi-nos.}

 

Para que Vos digneis, por intercessão de São Pantaleão, dai-nos a caridade ao nosso próximo, nós \textbf{Vos rogamos, ouvi-nos.}

 

Para que Vos digneis, por intercessão de São Vito, ensinai-nos o valor de nossa alma, nós \textbf{Vos rogamos, ouvi-nos.}

 

Para que Vos digneis, por intercessão de São Cristóvão, preservai-nos do pecado, nós \textbf{Vos rogamos, ouvi-nos.}

 

Para que Vos digneis, por intercessão de São Denis, dai-nos a tranqüilidade de consciência. nós \textbf{Vos rogamos, ouvi-nos.}

 

Para que Vos digneis, por intercessão de São Ciríaco, alcançai-nos a resignação a Vossa Santa Vontade, nós \textbf{Vos rogamos, ouvi-nos.}

 

Para que Vos digneis, por intercessão de Santo Eustáquio, dai-nos a paciência na adversidade, nós \textbf{Vos rogamos, ouvi-nos.}

 

Para que Vos digneis, por intercessão de Santo Acácio, alcançai-nos uma morte feliz, nós \textbf{Vos rogamos, ouvi-nos.}

 

Para que Vos digneis, por intercessão de Santo Egídio, alcançai-nos um julgamento misericordioso, nós \textbf{Vos rogamos, ouvi-nos.}

 

Para que Vos digneis, por intercessão de Santa Margarida, preservai-nos do inferno, nós \textbf{Vos rogamos, ouvi-nos.}

 

Para que Vos digneis, por intercessão de Santa Catarina, encurtai nosso purgatório, nós \textbf{Vos rogamos, ouvi-nos.}

 

Para que Vos digneis, por intercessão de Santa Bárbara, recebei-nos no céu, nós \textbf{Vos rogamos, ouvi-nos.}

 

Para que Vos digneis, por intercessão de todos os Santo Auxiliares, atendei nossas preces, nós \textbf{Vos rogamos, ouvi-nos.}

 

Cordeiro de DEUS que tirais o pecado do Mundo,\textbf{ perdoai-nos Senhor}

Cordeiro de DEUS que tirais o pecado do Mundo,\textbf{ ouvi-nos Senhor}

Cordeiro de DEUS que tirais o pecado do Mundo,\textbf{ tende piedade de nós, Senhor}.

\vspace{1cm}

\versicle Rogai por nós, Catorze Santos Auxiliares,

\response Para que sejamos dignos das promessas de Cristo.

 

\subsubsection*{Oremos}

 

Onipotente e eterno Deus, que concedeste extraordinárias graças e dons em Vossos Santos Jorge, Brás, Erasmo, Pantaleão, Vito, Cristóvão, Denis, Ciríaco, Eustáquio, Acácio, Egídio, Margarida, Catarina e Bárbara, e os glorificastes com milagres; nós vos suplicamos ouvi todos os pedidos de todos os que invocam sua intercessão. Por Cristo, nosso Senhor. Amém

% \begin{center}
%   \large
%   \textbf{Pai-Nosso, Ave-Maria e Glória.}
% \end{center}


\vfill

\begin{center}
\subsection*{Créditos}
\href{https://quemrezasesalva.com.br/devocao/ladainha-dos-catorze-santos-auxiliares}{Quem reza se salva}\\
\href{https://www.praymorenovenas.com/14-holy-helpers-novena}{Pray More Novenas}\\ 
\end{center}


\end{document}
