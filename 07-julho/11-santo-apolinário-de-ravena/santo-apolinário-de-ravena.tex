\documentclass[a4paper]{article}
\usepackage[utf8]{inputenc}
\usepackage[T1]{fontenc}
\usepackage{ragged2e}
\usepackage{caladea}
\usepackage{graphicx}
\usepackage{hyperref}
\usepackage{fancyhdr}
\usepackage{titlesec}

\author{Adaptado de publicação revisada}
\date{}

% Centralizar e definir fonte da section
\titleformat{\section}
  {\centering\Large\bfseries}
  {}{0pt}{}

% Centralizar e definir fonte da subsection
\titleformat{\subsection}
  {\centering\large\bfseries}
  {}{0pt}{}

% Rodapé com imagem e página
\pagestyle{fancy}
\fancyhf{}
\fancyhead[C]{\textit{Novena a Santo Apolinário de Ravena}}
\fancyfoot[LO, CE]{%
  \includegraphics[scale=0.2]{assets/cross.png}\quad
  \textit{\textbf{Santo Apolinário}, rogai por nós.}
}
\fancyfoot[R]{\thepage}

\renewcommand{\contentsname}{Sumário}


\makeatletter
\renewcommand\normalsize{%
   \@setfontsize\normalsize{14pt}{14pt}%
}
\renewcommand\large{%
   \@setfontsize\large{18pt}{18pt}%
}
\renewcommand\Large{%
   \@setfontsize\Large{20pt}{0pt}%
}
\renewcommand\LARGE{%
   \@setfontsize\LARGE{24pt}{24pt}%
}
\makeatother


\begin{document}


\tableofcontents
\thispagestyle{empty}

% --- Vida / Origem da Novena ---
\newpage
\section*{Vida / Origem da Novena}
\addcontentsline{toc}{section}{Vida / Origem da Novena}

Como acontece com quase todos os Santos do século I, não dispomos de muitas fontes históricas críveis sobre a vida de Santo Apolinário, primeiro Bispo de Ravena. Viveu no tempo do Império Bizantino do Oriente, mas, seu encontro com o apóstolo Pedro parece ter sido decisivo em sua vida. No entanto, algumas fontes datam a existência histórica deste Santo, no máximo por volta dos anos 150-200.

Apolinário era um jovem de grandes expectativas, que vivia em Antioquia com a sua família pagã. Certo dia, chegou alguém à sua cidade, que falava, de um modo novo, de amar uns aos outros, como Deus nos ama: era Pedro; suas palavras eram as mesmas de Jesus, o Filho de Deus, que ele tinha visto, com seus próprios olhos, morrer e, depois, ressuscitar para redimir a humanidade, e do qual havia recebido a missão de edificar a sua Igreja. Por isso, Pietro viajou por todo canto, chegando à Síria, por volta do ano 44.
Apolinário ficou profundamente tocado pelas suas palavras, a ponto de segui-lo até Roma. Dali, Pedro o enviou a Classe, perto de Ravena, onde a Marinha romana tinha uma base, com centenas de marinheiros, provenientes, sobretudo, de terras orientais. Algumas fontes afirmam também que ele chegou até a evangelizar a Trácia e Mesia, durante cerca de 3 anos.

Apolinário era astuto, aprendia rapidamente as coisas e, sobretudo, falava bem. Por isso, conseguiu levar muitos à fé cristã, obtendo a conversão de inteiras famílias. Por este motivo, Pedro confiou-lhe a edificação da Igreja de Ravena, da qual foi pastor, ou seja, primeiro Bispo.
Ao chegar à cidade, curou a esposa do tribuno. Mas, logo que as autoridades descobriram, lhe pediram para oferecer ídolos aos deuses. Apolinário recusou, dizendo que os ídolos eram feitos de ouro e prata, matérias preciosas que deviam ser oferecidos aos pobres. Por isso, foi brutalmente espancado. Apesar deste início difícil, governou aquela Igreja, por cerca de 30 anos, ficando famoso como "sacerdote" e "confessor", pelos quais ainda hoje é recordado.

Apolinário interpretou, perfeitamente, a missão pastoral de Bispo, chegando a levar muitas almas à fé. Assim sendo, claro, ficou na mira dos pagãos. Naquela época, reinava Vespasiano, no ano 70 d.C.. Os pagãos o intimaram a não fazer pregações, mas ele não obedeceu. Certo dia, ao voltar de uma visita a um leprosário, foi espancado, com tanta brutalidade, que quase morreu. Na verdade, morreu sete dias depois. No lugar do martírio, foi-lhe construída uma basílica – hoje chamada Santo Apolinário em Classe –, consagrada em 549.
O culto a este Santo difundiu-se rapidamente, para além dos confins da cidade, chegando até Roma, por intermédio dos Papas Símaco e Honório I, enquanto o rei dos Francos, Clodoveu, lhe dedicou uma igreja perto de Dijon.

As relíquias do Santo foram trasladadas para a cidade de Ravena, no século IX, e conservadas em uma igreja, que, desde então, foi chamada Basílica de Santo Apolinário Novo.


% --- Oração Inicial (opcional) ---
\newpage

  \section*{Novena a Santo Apolinário de Ravena}
    \addcontentsline{toc}{section}{Novena a Santo Apolinário de Ravena}
  \subsection*{Oração Inicial}
    \addcontentsline{toc}{subsection}{Oração Inicial}

Deus eterno e todo-poderoso, que destes a santo Apolinário a graça de lutar pela justiça até a morte, concedei-nos, por sua intercessão, suportar por vosso amor as adversidades e correr ao encontro de vós, que sois a nossa vida. Amém!

\begin{center}
  \textbf{\textit{Pai-Nosso, Ave-Maria, e Glória.}}
\end{center}

% --- Oração Final ---
  \subsection*{Oração Final}
    \addcontentsline{toc}{subsection}{Oração Final}

Dirija seus fiéis, Senhor, no caminho da salvação eterna,
que o bispo Santo Apolinário demonstrou com seu ensinamento e martírio,
e conceda, por sua intercessão,
que possamos perseverar de tal forma no cumprimento de seus mandamentos
que mereçamos ser coroados com ele.
Por Nosso Senhor, Jesus Cristo, vosso Filho, na unidade do Espírito Santo, pelos séculos dos séculos. Amém.
\newpage

\vfill

\centering 
\textbf{Créditos:}\\
\href{https://aleteia.org/daily-prayer/wednesday-july-20/}{Aleteia}\\
\href{https://oracoes.hi7.co/oracao-a-santo-apolinario---bispo-de-ravena-570fe9aa72199.html}{Hi7.co}\\
\href{https://santo.cancaonova.com/santo/santo-apolinario-bispo-de-ravena/}{Canção Nova}\\

\end{document}
